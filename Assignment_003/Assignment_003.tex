%
% @author   Shmish  "shmish90@gmail.com"
% @legal    MIT     "(c) Christopher Schmitt"
%


\documentclass{article}


%
% Document Imports
%

\usepackage{fancyhdr}
\usepackage{extramarks}
\usepackage{amsmath}
\usepackage{amssymb}
\usepackage{amsthm}
\usepackage{amsfonts}
\usepackage{color}
\usepackage{tikz}
\usepackage{algpseudocode}
\usepackage{algorithm}



%
% Document Configuration
%

\newcommand{\hwAuthor}{Christopher K. Schmitt}
\newcommand{\hwSubject}{CS 463}
\newcommand{\hwSection}{Section 01}
\newcommand{\hwSemester}{Spring 2021}
\newcommand{\hwAssignment}{Assignment 3}


%
% Document Environments
%

\setlength{\headheight}{65pt}
\pagestyle{fancy}
\lhead{\hwAuthor}
\rhead{
  \hwSubject \\
  \hwSection \\
  \hwSemester \\
  \hwAssignment
}

\newenvironment{problem}[1]{
  \nobreak\section*{Problem #1}
}{}

\newcommand*{\bigO}[1]{\ensuremath{\mathcal{O}\left(#1\right)}}
\newcommand*{\bigOmega}[1]{\ensuremath{\Omega\left(#1\right)}}
\newcommand*{\bigTheta}[1]{\ensuremath{\Theta\left(#1\right)}}


%
% Document Start
%

\begin{document}
  \begin{problem}{1}
    To compute $A \cup B$, we begin with an empty set, $C$.  We copy
    every element of $A$ into $C$.  This takes $\bigO{m}$ time.
    Next, we sort $A$ in $\bigO{m\log{m}}$ time.  This step will
    enable us to perform binary search on $A$ in $\bigO{\log{m}}$
    time.  For each element of $B$, perform binary search on $A$.
    If the element is not found in $A$, add it to $C$.  This takes
    $\bigO{n\log{m}}$ time, since we preform $n$ binary searches.
    This totals $m + m\log{m} + n\log{m}$, or simply \bigO{n\log{m}}.

    To compute $A \cap B$, we again start with an empty set, $C$.  We
    start by sorting $A$, again in \bigO{m\log{m}} time.  For every
    element of $B$, perform binary search on $A$.  This will total
    \bigO{n\log{m}} time.  If the element of $B$ is found in $A$, add
    it to $C$.  This will take a total of $m\log{m} + n\log{m}$ time,
    or just \bigO{n\log{m}}.
  \end{problem}

  \begin{problem}{2}
    Because the intervals are disjoint, we can define a strict
    ordering by sorting them either their upper or lower bounds.  We
    will choose to sort by the lower bound here.  This takes
    \bigO{q\log{q}} time.  Next, create an empty list of size $q$ and
    initialize each cell to zero.  This takes \bigO{q} time.  For
    each element of $X$, identify the interval it belongs to by the
    following method: select the center interval.  If the element of
    $X$ is in the center interval, increment the corresponding cell
    in the list by one.  If the element of $X$ is greater than the
    intervals upper bound, perform the same operation with endpoints
    at the current interval and last interval, recursively.  If the
    element of $X$ is less than the lower bound of the interval,
    perform the same operation with endpoints at the current interval
    and the first interval, recursively.  Because this is just a
    modified binary search, it runs in \bigO{\log{q}} time.  If we
    do this for every element of $X$, we will have a total of
    $q\log{q} + n\log{q}$ or just \bigO{n\log{q}}.
  \end{problem}

  \begin{problem}{3}
    Because the list can be pre-processed to contain a list of pairs
    in the form of (index, value).  Therefore, finding the maximum
    element and the index of the maximum element are the same thing.
    The following algorithm will Therefore simply identify the 
    maximum element of a list, as that is a cleaner solution.

    \begin{algorithm}[H]
      \caption{Return the maximum element of a list recursively}
      \begin{algorithmic}
        \Function{findMax}{list}
          \If{list.length = 1}
            \State \Return list[0]
          \Else
            \State lhs $\gets$ findMax(list[0 .. list.length / 2])
            \State rhs $\gets$ findMax(list[list.length / 2 ..])
            \State \Return max(lhs, rhs)
          \EndIf
        \EndFunction
      \end{algorithmic}
    \end{algorithm}

    This algorithm calls itself twice on an array of half the size
    in the recursive case.  We can establish the following recurrence
    relation.  Note that comparison is constant time.

    \begin{displaymath}
      T(n) = 2 \cdot T\left(\frac{n}{2}\right) + 1
    \end{displaymath}

    This is a trivial case one of the master theorem (because 
    $c = 1$ and $c < \log_2{2}$). By the master theorem, this
    algorithm belongs in \bigTheta{n^{\log_2{2}}}, or just
    \bigTheta{n}.  This is the same complexity as the brute-force
    method.
  \end{problem}
\end{document}
