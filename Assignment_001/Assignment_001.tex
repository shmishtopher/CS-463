%
% @author   Shmish  "shmish90@gmail.com"
% @legal    MIT     "(c) Christopher Schmitt"
%


\documentclass{article}


%
% Document Imports
%

\usepackage{fancyhdr}
\usepackage{extramarks}
\usepackage{amsmath}
\usepackage{amssymb}
\usepackage{amsthm}
\usepackage{amsfonts}
\usepackage{color}
\usepackage{tikz}



%
% Document Configuration
%

\newcommand{\hwAuthor}{Christopher K. Schmitt}
\newcommand{\hwSubject}{CS 463}
\newcommand{\hwSection}{Section 01}
\newcommand{\hwSemester}{Spring 2021}
\newcommand{\hwAssignment}{Assignment 1}


%
% Document Environments
%

\setlength{\headheight}{65pt}
\pagestyle{fancy}
\lhead{\hwAuthor}
\rhead{
  \hwSubject \\
  \hwSection \\
  \hwSemester \\
  \hwAssignment
}

\newenvironment{problem}[1]{
  \nobreak\section*{Problem #1}
}{}

\newcommand*{\bigO}[1]{\ensuremath{\mathcal{O}\left(#1\right)}}
\newcommand*{\bigOmega}[1]{\ensuremath{\Omega\left(#1\right)}}
\newcommand*{\bigTheta}[1]{\ensuremath{\Theta\left(#1\right)}}


%
% Document Start
%

\begin{document}
  \begin{problem}{1 (a)}
    \begin{proof}
      \begin{equation*}
        \begin{split}
          & 2n^2 + 3n + 7 \in \bigO{n^2} \\
          \implies & 2n^2 + 3n + 7 \le c \cdot n^2 \\
          \implies & \frac{2n^2}{n^2} + \frac{3n}{n^2} + \frac{7}{n^2} \le c \\
          \implies & 2 + \frac{3}{n} + \frac{7}{n^2} \le c
        \end{split}
      \end{equation*}
      The Big-Oh condition holds for $n \ge 1$ when $c = 12$.  Therefore $2n^2 + 3n + 7 \in \bigO{n^2}$. 
    \end{proof}
  \end{problem}

  \begin{problem}{1 (b)}
    \begin{proof}
      \begin{equation*}
        \begin{split}
          & 100n^3 - n^2 + 5n \in \bigO{n^3} \\
          \implies & 100n^3 - n^2 + 5n \le c \cdot n^3 \\
          \implies & \frac{100n^3}{n^3} - \frac{n^2}{n^3} + \frac{5n}{n^3} \le c \\
          \implies & 100 - \frac{1}{n} + \frac{5}{n^2} \le c
        \end{split}
      \end{equation*}
      The Big-Oh condition holds for $n \ge 1$ when $c = 104$.  Therefore $100n^3 - n^2 + 5n \in \bigO{n^3}$.
    \end{proof}
  \end{problem}

  \begin{problem}{1 (c)}
    \begin{proof}
      \begin{equation*}
        \begin{split}
          & 15n^4 + 3n^3 \in \bigOmega{n^4} \\
          \implies & 15n^4 + 3n^3 \ge c \cdot n^4 \\
          \implies & \frac{15n^4}{n^4} + \frac{3n^3}{n^4} \ge c \\
          \implies & 15 + \frac{3}{n} \ge c
        \end{split}
      \end{equation*}
      The Big-Omega condition holds for $n \ge 1$ when $c = 18$.  Therefore $15n^4 + 3n^3 \in \bigOmega{n^4}$.
    \end{proof}
  \end{problem}

  \begin{problem}{1 (d)}
    \begin{proof}
      \begin{equation*}
        \begin{split}
          & 2n^2\log{n} - 2n^2 \in \bigOmega{n^2} \\
          \implies & 2n^2\log{n} - 2n^2 \ge c \cdot n^2 \\
          \implies & \frac{2n^2\log{n}}{n^2} - \frac{2n^2}{n^2} \ge c \\
          \implies & 2\log{n} - 2 \ge c
        \end{split}
      \end{equation*}
      The Big-Omega condition holds for $n \ge 4$ when $c = 2$.  Therefore $2n^2\log{n} - 2n^2 \in \bigOmega{n^2}$.
    \end{proof}
  \end{problem}

  \begin{problem}{1 (e)}
    \begin{proof}
      \begin{equation*}
        \begin{split}
          & a_{k}n^{k} + a_{k-1}n^{k-1} + \dotsb + a_0 \in \bigTheta{n^k} \\
          \implies & a_{k}n^{k} + a_{k-1}n^{k-1} + \dotsb + a_0 \le c_0 \cdot n^k \\
          \implies & \frac{a_{k}n^{k}}{n^k} + \frac{a_{k-1}n^{k-1}}{n^k} + \dotsb + \frac{a_0}{n^k} \le c_0 \\
          \implies & a_k + \frac{a_{k-1}}{n^1} + \frac{a_{k-2}}{n^2} + \dotsb + \frac{a_0}{n^{k}} \le c_0 \\
          & a_{k}n^{k} + a_{k-1}n^{k-1} + \dotsb + a_0 \in \bigTheta{n^k} \\
          \implies & a_{k}n^{k} + a_{k-1}n^{k-1} + \dotsb + a_0 \ge c_1 \cdot n^k \\
          \implies & \frac{a_{k}n^{k}}{n^k} + \frac{a_{k-1}n^{k-1}}{n^k} + \dotsb + \frac{a_0}{n^k} \ge c_1 \\
          \implies & a_k + \frac{a_{k-1}}{n^1} + \frac{a_{k-2}}{n^2} + \dotsb + \frac{a_0}{n^{k}} \ge c_1
        \end{split}
      \end{equation*}
      The Big-Theta condition holds for $n \ge 1$ when $c_0 = c_1 = \Sigma_{i = 0}^{k} a_i$.  Therefore $a_{k}n^{k} + a_{k-1}n^{k-1} + \dotsb + a_0 \in \bigTheta{n^k}$.
    \end{proof}
  \end{problem}

  \begin{problem}{2 (a)}
    \begin{proof}
      \begin{equation*}
          f(n) \in \bigO{g(n)} \implies f(n) \le c \cdot g(n) \implies \frac{1}{c} \cdot f(n) \le g(n)
      \end{equation*}
      Let $d = \frac{1}{c}$.  We can then rewrite the above expression as:
      \begin{equation*}
        g(n) \ge d \cdot f(n) \implies g(n) \in \bigOmega{f(n)}
      \end{equation*} 
    \end{proof}
  \end{problem}

  \begin{problem}{2 (b)}
    Since both functions are positive, $max \{f(n), g(n)\}$ is surely less than
    $f(n) + g(n)$.  It must also be true that $2 \cdot max \{f(n), g(n)\}$ is greater
    than $f(n) + g(n)$.  This gives us $max \{f(n), g(n)\} \le f(n) + g(n) \le 2 \cdot max \{f(n), g(n)\}$.
    Because $f(n) + g(n)$ is bounded on the top and bottom by constants, the Big-Theta condition is satisfied.
  \end{problem}

  \begin{problem}{3}
    \begin{proof}
      \begin{equation*}
        \begin{split}
          f_1(n) + f_2(n) & \le c_0 \cdot g_1(n) + c_1 \cdot g_2(n) \\
          & \le (c_0 + c_1) \cdot max(g_1(n) + g_2(n))
        \end{split}
      \end{equation*}

      \begin{equation*}
        \begin{split}
          f_1(n) + f_2(n) & \ge c_2 \cdot g_1(n) + c_3 \cdot g_2(n) \\
          & \ge (c_2 + c_3) \cdot max(g_1(n) + g_2(n))
        \end{split}
      \end{equation*}
    \end{proof}
  \end{problem}

  \begin{problem}{4 (a)}
    \begin{center}
      \bigTheta{n^{12}}
    \end{center}

    \begin{proof}
      \begin{equation*}
        \lim_{n\to\infty} \frac{(2n + 15)^{12}}{n^{12}} = 4096
      \end{equation*}
    \end{proof}
  \end{problem}

  \begin{problem}{4 (b)}
    \begin{center}
      \bigTheta{n^{\frac{3}{2}}}
    \end{center}
    
    \begin{proof}
      \begin{equation*}
        \lim_{n\to\infty} \frac{\sqrt{4n^3 + 3n^2 + 1}}{n^{\frac{3}{2}}} = 2
      \end{equation*}
    \end{proof}
  \end{problem}

  \begin{problem}{4 (c)}
    \begin{center}
      \bigTheta{n^2\log{n}}
    \end{center}

    \begin{proof}
      \begin{equation*}
        \lim_{n\to\infty} \frac{(n + 2)^2\log{n/2}}{n^2\log{n}} = 1
      \end{equation*}
      The first term of the expression is lower order, so can be safely
      ignored as it will approach zero as $n \to \infty$.
    \end{proof}
  \end{problem}

  \begin{problem}{4 (d)}
    \begin{center}
      \bigTheta{2^{2n}}
    \end{center}

    \begin{proof}
      \begin{equation*}
        \lim_{n\to\infty} \frac{2^{2n+1}+3^{n-10}}{2^{2n}} = 2
      \end{equation*}
    \end{proof}
  \end{problem}
\end{document}
