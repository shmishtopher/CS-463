%
% @author   Shmish  "shmish90@gmail.com"
% @legal    MIT     "(c) Christopher Schmitt"
%


\documentclass{article}


%
% Document Imports
%

\usepackage{fancyhdr}
\usepackage{extramarks}
\usepackage{amsmath}
\usepackage{amssymb}
\usepackage{amsthm}
\usepackage{amsfonts}
\usepackage{color}
\usepackage{tikz}
\usepackage{algpseudocode}
\usepackage{algorithm}



%
% Document Configuration
%

\newcommand{\hwAuthor}{Christopher K. Schmitt}
\newcommand{\hwSubject}{CS 463}
\newcommand{\hwSection}{Section 01}
\newcommand{\hwSemester}{Spring 2021}
\newcommand{\hwAssignment}{Assignment 5}


%
% Document Environments
%

\setlength{\headheight}{65pt}
\pagestyle{fancy}
\lhead{\hwAuthor}
\rhead{
  \hwSubject \\
  \hwSection \\
  \hwSemester \\
  \hwAssignment
}

\newenvironment{problem}[1]{
  \nobreak\section*{Problem #1}
}{}

\newcommand*{\Let}[2]{\State #1 $\gets$ #2}
\newcommand*{\bigO}[1]{\ensuremath{\mathcal{O}\left(#1\right)}}
\newcommand*{\bigOmega}[1]{\ensuremath{\Omega\left(#1\right)}}
\newcommand*{\bigTheta}[1]{\ensuremath{\Theta\left(#1\right)}}


%
% Document Start
%

\begin{document}
  \begin{problem}{1}
    In order to maximize profit, we can create a simple recursive
    function that makes a cut at some position, $k$.  We are then
    left with a rod of length $n - k$.  We recursively cut the rod
    at every possible position, returning the optimal solution at
    the end.  This approach is clearly exponential time, but
    involves re-computing potions of the pipe at each step.  We
    can eliminate this with a "bottom-up" approach.  We can create
    a temporary array to store the solutions of sub-problems and
    avoid re-computing them.  We start by solving for the smaller
    rods, as they will be used to create our solution.

    \begin{algorithm}
      \caption{Rod Cutting}
      \begin{algorithmic}[1]
        \Function{cut}{$cost$, $n$}
          \Let{$temp$}{$[0, \dotso, n]$}
          \For{$i = 1$ to $n$}
            \Let{$x$}{$-\infty$}
            \For{$j = 1$ to $i$}
              \Let{$x$}{MAX($x, cost[j] + temp[i - j]$)}
            \EndFor
            \Let{$temp[i]$}{$x$}
          \EndFor
          \State \Return $temp[n]$
        \EndFunction
      \end{algorithmic}
    \end{algorithm}

    This algorithm belongs to \bigO{n^2} because the MAX function
    and the innermost addition are both constant time.  We can
    therefore solve for the running-time with the following 
    sigma-expression.

    \begin{displaymath}
      \sum_{i = 1}^{n} \sum_{j = 1}^{i} 1 = \frac{n^2 + n}{2} \in \bigO{n^2}
    \end{displaymath}

    The top-down approach (memoization) has identical performance
    characteristics, but uses a recursive method instead of an
    iterative one.
  \end{problem}

  \pagebreak

  \begin{problem}{2}
    To compute the number of unique paths, we can use a naïve 
    recursive algorithm to solve this problem in exponential time.  
    This would involve re-computing the number of unique paths in
    identical sub-grids several times.  To eliminate this, we can
    store the number of unique paths to get from any point to the
    destination in an $m \times n$ matrix.  We can then use a
    bottom-up approach to fill the matrix, starting from the 
    destination and iterating up to the start.

    \begin{algorithm}
      \caption{Unique Paths}
      \begin{algorithmic}[1]
        \Function{paths}{$m, n$}
          \Let{temp}{$\begin{pmatrix}
            0_{0,0} & 0_{0,1} & \cdots & 0_{0,n-1} \\
            0_{1,0} & 0_{1,1} & \cdots & 0_{1,n-1} \\
            \vdots  & \vdots  & \ddots & \vdots  \\
            0_{m-1,0} & 0_{m-1,1} & \cdots & 0_{m-1,n-1} 
            \end{pmatrix}$}

          \For{$i = 0$ to $m$}
            \Let{temp[$i$][0]}{1}
          \EndFor

          \For{$i = 0$ to $n$}
            \Let{temp[0][$i$]}{1}
          \EndFor

          \For{$i = 1$ to $m$}
            \For{$j = 0$ to $n$}
              \Let{temp[$i$][$j$]}{temp[$i - 1$][$j$] + temp[$i$][$j - 1$]}
            \EndFor
          \EndFor
          \State \Return temp[$m - 1$][$n - 1$]
        \EndFunction
      \end{algorithmic}
    \end{algorithm}

    This algorithm belongs to \bigO{n \cdot m}, which we can see by solving
    the following sigma expression.

    \begin{displaymath}
      m + n + \sum_{i = 1}^{m} \sum_{j = 0}^{n} 1 = m(n + 1) + m + n = nm + 2m + n \in \bigO{nm}
    \end{displaymath}

    It should that this relation only holds as long as $n \ge 2$.
    Using combinatorics, we can find an even cleaner solution.
    We can use the formula $\binom{m + n - 2}{n - 1}$.  Subbing in
    $5$ and $6$ for $m$ and $n$ gives us $\binom{9}{5} = 126$.
  \end{problem}

  \pagebreak

  \begin{problem}{3}
    Suppose we create a function, $f(i, j)$, that return the maximum
    profit path from the bottom to row to $(i, j)$.  This function 
    will use the value $p(i, j)$ plus the result from the previous
    row.  In this manner, we can avoid re-computing previous rows by
    filling a matrix in a bottom-up manner.  A simpler solution 
    however, would be to memoize the recursive definition.  We can
    construct a recursive definition like so.

    \begin{displaymath}
      f(i, j) = \begin{cases}
        -\infty & j < 0 \\
        -\infty & j \ge n \\
        p(i, j) & i = 0 \\
        max(f(i - 1, j - 1), f(i - 1, j), f(i, j - 1)) + p(i, j) & otherwise
      \end{cases}
    \end{displaymath}

    By using the definition, we can see that this is an exponential
    time computation.  However, memoization will mean that we only
    check each ordered pair once.  On an $n \times n$ board, this
    would put us at \bigO{n^2} complexity.  To construct the path, we
    can create an algorithm like this.

    \begin{algorithm}
      \caption{Checkerboard}
      \begin{algorithmic}[1]
        \Function{construct}{$i, j, memo, path$}
          \If{i = 0}
            \State \Return path
          \EndIf
          \Let{a}{memo[i - 1][j - 1] or construct(i - 1, j - 1, memo, path)}
          \Let{b}{memo[i - 1][j] or construct(i - 1, j, memo, path)}
          \Let{c}{memo[i][j - 1] or construct(i, j - 1, memo, path)}
          \Let{memo[i][j]}{max(a, b, c) + p(i, j)}
          \State path.push(j)
          \State \Return construct(i - 1, j, memo, path)
        \EndFunction
      \end{algorithmic}
    \end{algorithm}
  \end{problem}

  \pagebreak

  \begin{problem}{4}
    We can either include or exclude each element in the knapsack.
    We could use a simple recursive solution to explore every 
    possible combination of includes.  Without memoization, this
    would involve re-computing many include-exclude combinations.  We
    can instead build a table in a bottom-up manner to avoid this
    needless work.

    Our algorithm will begin by populating a table with the base
    case.  Since we cannot add items when we have none to choose
    from, the solution for the first row is always zero.  At each
    row, $i$, we will compute the maximum value we could get by
    including the the $i$'th element.  This will be equal to 
    $temp[i-1][j] + value[i]$.

    \begin{algorithm}
      \caption{knapsack}
      \begin{algorithmic}[1]
        \Function{knapsack}{$n, w, weight, value$}
          \Let{temp}{$\begin{pmatrix}
            0_{0,0} & 0_{0,1} & \cdots & 0_{0,1} \\
            0_{1,0} & 0_{1,1} & \cdots & 0_{1,n} \\
            \vdots  & \vdots  & \ddots & \vdots  \\
            0_{w,0} & 0_{w,1} & \cdots & 0_{w,n} 
            \end{pmatrix}$}

          \For{$i = 1$ to $n$}
            \For{$j = 1$ to $w$}
              \Let{max exclude}{temp[$i - 1$][$j$]}
              \Let{max include}{$-\infty$}
              \If{weight[$i$] $<$ $j$} \Comment{If the knapsack can fit the item}
                \Let{max include}{value[$i$] $+$ temp[i][j - weight[$i$]]}
              \EndIf

              \Let{temp[i][j]}{max(max include, max exclude)}
            \EndFor
          \EndFor

          \State \Return temp[$n$][$w$]
        \EndFunction
      \end{algorithmic}
    \end{algorithm}

    This algorithm fills every entry in a $|value| \times |weight|$ 
    table.  If the size of these collections are $n$ and $w$ 
    respectively, then the complexity of this algorithm is \bigO{nw}.
    With the given conditions, the maximum knapsack value is 75.
  \end{problem}
\end{document}