%
% @author   Shmish  "shmish90@gmail.com"
% @legal    MIT     "(c) Christopher Schmitt"
%


\documentclass{article}


%
% Document Imports
%

\usepackage{fancyhdr}
\usepackage{extramarks}
\usepackage{amsmath}
\usepackage{amssymb}
\usepackage{amsthm}
\usepackage{amsfonts}
\usepackage{color}
\usepackage{tikz}
\usepackage{tikz-qtree}
\usepackage{algpseudocode}
\usepackage{algorithm}



%
% Document Configuration
%

\newcommand{\hwAuthor}{Christopher K. Schmitt}
\newcommand{\hwSubject}{CS 463}
\newcommand{\hwSection}{Section 01}
\newcommand{\hwSemester}{Spring 2021}
\newcommand{\hwAssignment}{Assignment 6}


%
% Document Environments
%

\setlength{\headheight}{65pt}
\pagestyle{fancy}
\lhead{\hwAuthor}
\rhead{
  \hwSubject \\
  \hwSection \\
  \hwSemester \\
  \hwAssignment
}

\newenvironment{problem}[1]{
  \nobreak\section*{Problem #1}
}{}

\newcommand*{\Let}[2]{\State #1 $\gets$ #2}
\newcommand*{\bigO}[1]{\ensuremath{\mathcal{O}\left(#1\right)}}
\newcommand*{\bigOmega}[1]{\ensuremath{\Omega\left(#1\right)}}
\newcommand*{\bigTheta}[1]{\ensuremath{\Theta\left(#1\right)}}


%
% Document Start
%

\begin{document}
  \begin{problem}{1}
    We can use a greedy approach to develop an efficient solution to
    the fractional knapsack problem.  To determine the most 
    profitable configuration, we will start by computing the ratio
    of the value to the weight for each item.  We can then sort items
    by their value to weight ration and pull items out of the list
    and into our knapsack until we reach the weight limit.  For the
    final item in the knapsack, we may need to place a fractional
    item.  This is not the case for the other items, as it would not
    make sense to include a fraction of a more valuable item until
    we either run out of space or consume the entire item.

    \begin{algorithm}
      \caption{Fractional Knapsack}
      \begin{algorithmic}[1]
        \Function{knapsack}{weights, values, capacity}
          \Let{items}{$(r_i : r_i = values[i] / weights[i])$}
          \Let{knapsack}{empty set}

          \While{$capacity > 0$}
            \Let{$maximum_i$}{$max(items)$}
            \If{$capacity \ge weights[i]$}
              \State knapsack.push($values[i]$)
              \Let{$capacity$}{$capacity - weights[i]$}
              \State remove items[$i$]
              \State remove weights[$i$]
              \State remove values[$i$]
            \Else
              \Let{ratio}{$capacity / weight[i]$}
              \State knapsack.push($radio \cdot values[i]$)
              \Let{$capacity$}{$0$}
            \EndIf
          \EndWhile
          \State \Return $knapsack$
        \EndFunction
      \end{algorithmic}
    \end{algorithm}
  \end{problem}

  \pagebreak

  \begin{problem}{2}
    \begin{center}
      \begin{tabular}{||c c c c||} 
      \hline
      Step & $V$ & $MST$ & Weights \\
      \hline\hline
      0 & $\{1, 2, 3, 4, 5\}$ & $\{\}$ & $(0, \infty, \infty, \infty, \infty)$ \\ 
      \hline
      1 & $\{2, 3, 4, 5\}$ & $\{1\}$ & $(0, 11, 4, 2, \infty)$ \\ 
      \hline
      2 & $\{2, 3, 5\}$ & $\{1, 4\}$ & $(0, 11, 4, 2, 7)$ \\
      \hline
      3 & $\{2, 5\}$ & $\{1, 4, 3\}$ & $(0, 11, 4, 2, 5)$ \\
      \hline
      4 & $\{2\}$ & $\{1, 4, 3, 5\}$ & $(0, 11, 4, 2, 7)$ \\
      \hline
      5 & $\{\}$ & $\{1, 4, 3, 5, 2\}$ & $(0, 11, 4, 2, 7)$ \\
      \hline
     \end{tabular}
    \end{center}

    \begin{center}
      \begin{tikzpicture}
        \path node[draw, circle] (1) at ({360 / 5 * (1 - 1)}:3) {$1$};
        \path node[draw, circle] (2) at ({360 / 5 * (1 - 2)}:3) {$2$};
        \path node[draw, circle] (3) at ({360 / 5 * (1 - 3)}:3) {$3$};
        \path node[draw, circle] (4) at ({360 / 5 * (1 - 4)}:3) {$4$};
        \path node[draw, circle] (5) at ({360 / 5 * (1 - 5)}:3) {$5$};

        \draw (1) -- (4);
        \draw (4) -- (3);
        \draw (4) -- (2);
        \draw (3) -- (5);
      \end{tikzpicture}
    \end{center}
  \end{problem}

  \pagebreak

  \begin{problem}{3}
    \begin{center}
      \begin{tikzpicture}
        \Tree [.{$0.06$}
          {$(f, 0.02)$}
          {$(g, 0.04)$}
        ]
      \end{tikzpicture}
    \end{center}

    \begin{center}
      \begin{tikzpicture}
        \Tree [.{$0.16$}
          [.{$0.06$}
            {$(f, 0.02)$}
            {$(g, 0.04)$}
          ]
          [.{$0.10$}
            {$(a, 0.05)$}
            {$(c, 0.05)$}
          ]
        ]
      \end{tikzpicture}
    \end{center}

    \begin{center}
      \begin{tikzpicture}
        \Tree [.{$1.00$}
          [.{$0.84$}
            {$(e, 0.45)$}
            [.{$0.39$}
              {$(d, 0.20)$}
              {$(b, 0.19)$}
            ]
          ]
          [.{$0.16$}
            [.{$0.06$}
              {$(f, 0.02)$} 
              {$(g, 0.04)$}
            ]
            [.{$0.10$} 
              {$(a, 0.05)$}
              {$(c, 0.05)$}
            ]
          ]
        ]
      \end{tikzpicture}
    \end{center}

    \begin{center}
      \begin{tabular}{||c c c||} 
      \hline
      Symbol & Code & Frequency \\
      \hline\hline
      a & 001 & 0.05 \\ 
      \hline
      b & 100 & 0.19 \\ 
      \hline
      c & 000 & 0.05 \\ 
      \hline
      d & 101 & 0.20 \\ 
      \hline
      e & 11 & 0.45 \\ 
      \hline
      f & 011 & 0.02 \\ 
      \hline
      g & 010 & 0.04 \\ 
      \hline
     \end{tabular}
    \end{center}

    \begin{displaymath}
      \mathcal{L} = 3 \cdot 0.05 + 3 \cdot 0.19 + 3 \cdot 0.05 + 3 \cdot 0.20 + 2 \cdot 0.45 + 3 \cdot 0.02 + 3 \cdot 0.04 = 2.55
    \end{displaymath}
  \end{problem}

  \pagebreak

  \begin{problem}{4}
    \begin{center}
      \begin{tabular}{||c c c c c c||}
        \hline
        Round & m1 & m2 & m3 & m4 & m5 \\
        \hline\hline
        0 & | & | & | & | & | \\
        \hline
        1 & w3 & w4 & | & w1 & w5 \\
        \hline
        2 & w3 & | & w1 & w4 & w5 \\
        \hline
        3 & w3 & | & w1 & w4 & w5 \\
        \hline
        4 & w3 & w1 & | & w4 & w5 \\
        \hline
        5 & w3 & w1 & | & w4 & w5 \\
        \hline
        6 & w3 & w1 & w2 & w4 & w5 \\
        \hline
      \end{tabular}
    \end{center}

    \begin{center}
      Results
    \end{center}

    \begin{center}
      \begin{tabular}{||c c||}
        \hline
        $M_i$ & $W_i$ \\
        \hline\hline
        1 & 3 \\
        \hline
        2 & 1 \\
        \hline
        3 & 2 \\
        \hline
        4 & 4 \\
        \hline
        5 & 5 \\
        \hline
      \end{tabular}
    \end{center}
  \end{problem}
\end{document}